\documentclass[12pt,reqno, twoside]{amsbook}
%%%%%%%%%%%%%%%%%%%%%%%%%%%%%%%%%%%%%%%%%%%%%%%%%%%%%%%%%%%%%%%%%%%%%%%%%%%%%%%%%%%%%%%%%%%%%%%%%%%%%%%%%%%%%%%%%%%%%%%%%%%%%%%%%%%%%%%%%%%%%%%%%%%%%%%%%%%%%%%%%%%%%%%%%%%%%%%%%%%%%%%%%%%%%%%%%%%%%%%%%%%%%%%%%%%%%%%%%%%%%%%%%%%%%%%%%%%%%%%%%%%%%%%%%%%%
\usepackage{eurosym}
\usepackage{amsmath}
\usepackage{amssymb}
\usepackage{amsfonts}
\usepackage[onehalfspacing]{setspace}
\usepackage{chngcntr}
\usepackage{graphicx}
\usepackage[a4paper, margin=2.5cm]{geometry}
\usepackage[english]{babel}
\usepackage{fancyhdr}
\usepackage{titlesec}
\usepackage{enumitem}
\usepackage{etoolbox}
\usepackage{comment}
\usepackage{caption}
% you may need to uncomment the command below if you use eps format for figures
%\usepackage{epstopdf}

\makeatletter
\def
\section{\@startsection{section}{1}%
  \z@{.5\linespacing\@plus.7\linespacing}{.25\linespacing}%
{\normalfont\bfseries\flushleft}}
\def
\subsection{\@startsection{subsection}{2}%
  \z@{.5\linespacing\@plus.7\linespacing}{.25\linespacing}%
{\normalfont\bfseries\flushleft}}

\makeatother
\setcounter{MaxMatrixCols}{10}

\providecommand{\U}[1]{\protect\rule{.1in}{.1in}}
\theoremstyle{plain}
\newtheorem{acknowledgement}{Acknowledgement}
\newtheorem{algorithm}{Algorithm}[chapter]
\newtheorem{axiom}{Axiom}[chapter]
\newtheorem{case}{Case}[chapter]
\newtheorem{claim}{Claim}[chapter]
\newtheorem{conclusion}{Conclusion}[chapter]
\newtheorem{condition}{Condition}[chapter]
\newtheorem{conjecture}{Conjecture}[chapter]
\newtheorem{corollary}{Corollary}[chapter]
\newtheorem{criterion}{Criterion}[chapter]
\newtheorem{definition}{Definition}[chapter]
\newtheorem{example}{Example}[chapter]
\newtheorem{exercise}{Exercise}[chapter]
\newtheorem{lemma}{Lemma}[chapter]
\newtheorem{notation}{Notation}[chapter]
\newtheorem{problem}{Problem}[chapter]
\newtheorem{proposition}{Proposition}[chapter]
\newtheorem{remark}{Remark}[chapter]
\newtheorem{solution}{Solution}[chapter]
\newtheorem{summary}{Summary}[chapter]
\newtheorem{theorem}{Theorem}[chapter]
\numberwithin{equation}{chapter}

% Please write the references according to your school

\newenvironment{dedication}
{%\clearpage           % we want a new page
  \thispagestyle{empty}% no header and footer
  \vspace*{\stretch{1}}% some space at the top
  \itshape             % the text is in italics
  \raggedleft          % flush to the right margin
}
{\par % end the paragraph
  \vspace{\stretch{3}} % space at bottom is three times that at the top
  %\clearpage           % finish off the page
}
\numberwithin{section}{chapter}
\fancyhead{}
\fancyfoot{}
\pagestyle{fancy}
\fancyfoot[LE,RO]{\thepage}

\makeatletter
\def\ps@plain{\ps@empty
  \def\@evenfoot{%
    \normalfont\scriptsize
    \rlap{\thepage}\hfil
  }%
  \def\@oddfoot{%
    \normalfont\scriptsize \hfil
  \llap{\thepage}}%
}
\makeatother
\renewcommand{\headrulewidth}{0pt}
\renewcommand{\footrulewidth}{0pt}

\begin{document}
\frontmatter
\addtocontents{toc}{\setcounter{tocdepth}{2}}
\thispagestyle{empty}

\begin{dedication}%

  \begin{flushright}
    \textit{Write here your dedication!}
  \end{flushright}%
\end{dedication}

\setcounter{page}{1}
\pagenumbering{roman} %
\chapter*{Cover}

\chapter*{Acknowledgment}

Write here the acknowledgments and grants, if any

\chapter*{Resumo}

Write here your abstract in Portuguese

\chapter*{Abstract}

Write here your abstract in English

\tableofcontents

\mainmatter
\setcounter{page}{1} \pagenumbering{arabic}

\chapter{Introduction}
\noindent The digital transformation of educational institutions has created new opportunities for integrating analytics and gamification in order to improve learning outcomes.
The Learning Scorecard (LS) platform embodies this approach, merging business intelligence with educational strategies to monitor and enhance student engagement and performance.

This thesis presents the development of the Learning Scorecard v2.5: a
prototype plugin that integrates some of the major features of the previous LS
iterations with the Moodle Learning Management System Service API.\@
\section{Background and Rationale}
\section{Statement of the Problem}
\section{Aims and Objectives}
\section{Methodological Approach}
\section{Contribution to Knowledge}
\section{Structure of the Thesis}

\chapter{Theoretical and Methodological Foundations}
\section{Gamification and Motivation in Education}
\subsection{Theoretical Perspectives on Gamification}
\subsection{Gamification and Student Engagement}
\section{Learning Analytics and Educational Data Mining}
\subsection{Defining Learning Analytics}
\subsection{Data-Driven Decision Making in Higher Education}

\chapter{The Learning Scorecard: Concept and Evolution}
\section{Genesis and Conceptualization of the Learning Scorecard}
\section{Evolution Through Iterative Development}
\subsection{Historical Overview of Versions}
\subsection{Lessons Learned from Prior Iterations}
\section{LS Concepts}
\noindent Throughout the development of the Learning Scorecard a lot of concepts have been introduced into the domain, some with a more clear mapping to a Moodle concept than others.
The best way to divide the LS concepts would be through each systems' main intended purpose.

\subsection{Academic: Institutional Structure}
\subsubsection{Course}
\paragraph\ A Course describes the overarching academic program (e.g., ``Computer Science``) that encompasses multiple curricular units. It also refers to the primary organizational division of a students degree.

\subsubsection{Curricular Unit}
\paragraph\ A Curricular Unit represents an individual subject within a Course, corresponding to specific learning objectives, content and assessments that students must complete as part of their academic progression.

\subsection{Academic: Human Resources (TBC)}
\subsubsection{Student}
\paragraph\ A Student is a user enrolled in one or more Curricular Units and in at least one Course who participates in learning activities, completes quests, earns experience points progresses through the gamified learning system.

\subsubsection{Teacher / Faculty}
\paragraph\ Teachers or Faculty members are users responsible for creating and managing curricular content, designing quests and monitoring student progress within their assigned Curricular Units.

\subsection{Academic: Curriculum}
\subsubsection{Syllabus Contents}
\paragraph\ Syllabus Contents encompass the planning of the Curricular Unit and all learning materials, resources and educational content within.

\subsection{Academic: Organizational}
\subsubsection{Calendar}
\paragraph\ The Calendar is a concept that is meant as an organizational system for both students and faculty to organize their quests and allow a better monthly planning.

\subsubsection{Timeline}
\paragraph\ The Timeline is another concept of the LS organizational system that provides a chronological summary of the quests from the Curricular Unit.

\subsection{Gamification: Core Progression}

\subsubsection{Experience Points (XP)}
\paragraph\ Experience Points serve as the fundamental display of progress within the Learning Scorecard platform.
Students earn XP through completing various educational activities, creating a quantifiable measure of academic engagement and achievement.
The XP system provides immediate feedback and enables comparison between students through leaderboards and progression tracking.

\subsubsection{Ranks}
\paragraph\ The rank system establishes hierarchical progression levels that students advance through based on accumulated XP.\@
Faculty members can configure rank thresholds dynamically, setting minimum and maximum XP values for each rank level.
The system includes ranks such as Newbie (entry level at 0 XP), Rookie, Skilled Expert, Master, and Legendary, with each rank unlocking new privileges and recognition within the platform.

\subsubsection{Quests}
\paragraph\ Quests represent the core educational activities that students complete to earn XP and demonstrate learning.
The system categorizes quests into five distinct types: Class Attendance, Practical Assignment, Quiz, Exercise, and Event.
Each quest can be designated as mandatory (milestone quests) or optional, allowing for flexible curriculum design.
Quests serve as the primary mechanism for connecting educational content with gamification elements.

\subsection{Gamification: Social Learning}

\subsubsection{Alliances}
\paragraph\ Alliances function as large-scale organizational units that typically correspond to academic classes, for example LEI or ETI.\@
These structures facilitate competition and comparison at the class level while maintaining educational coherence.
Students within alliances can compare their progress through dedicated leaderboards and participate in alliance-specific activities.

\subsubsection{Guilds}
\paragraph\ Guilds represent smaller collaborative units within alliances, typically corresponding to project teams or study groups.
The guild system promotes cooperative learning by encouraging mutual assistance among members.
Guild-specific achievements, badges, and leaderboards foster team spirit while maintaining individual accountability within the collaborative framework.

\subsection{Gamification: Achievement and Recognition}

\subsubsection{Badges}
\paragraph\ The badge system provides targeted recognition for specific accomplishments and behaviors.
The platform includes 39 different badges organized into four categories: individual achievements, guild-based accomplishments, forum participation, and final questionnaire completion.
Each badge category features multiple tiers (bronze, silver, gold, and platinum) that correspond to increasing levels of difficulty and commitment.

\subsubsection{Trophies}
\paragraph\ Trophies represent prestigious achievements awarded to top performers in various leaderboard categories.
The system awards trophies for first-place positions in overall rankings (Best Score Player), guild performance (Best Guild), and combined metrics (Best Triathlon Player).
Each trophy provides substantial XP rewards (typically 1000 XP) and serves as a highly visible symbol of excellence.

\subsubsection{Avatars}
\paragraph\ The avatar system enables profile customization while serving as a visual indicator of progression.
Students unlock new avatar options as they advance through rank levels, with both male and female options available at each tier.
This customization feature allows for personal expression while maintaining the connection between visual representation and academic achievement.

\subsection{Gamification: Competitive and Motivation}

\subsubsection{Leaderboards}
\paragraph\ Multiple leaderboard systems enable various forms of competition and comparison.
The platform features five distinct leaderboards: overall individual rankings, guild-based comparisons, exercise-specific performance, quiz results, and combined metrics.
Each leaderboard displays position, username, rank, avatar, and XP totals, creating transparency in academic performance while fostering healthy competition.

\subsubsection{Last Chances}
\paragraph\ The Last Chance mechanism provides opportunities for recovery and continued engagement when students fall behind or make mistakes.
This system prevents permanent failure states that could lead to disengagement, ensuring that all students can continue participating and improving their performance throughout the academic term.

\begin{comment}
\subsection{Monitoring and Analytics}

\subsubsection{Student Dashboard}
\paragraph\ The student dashboard serves as the central interface for progress monitoring and decision-making.
Students can view their advancement across different quest types, track XP accumulation, monitor rank progression, and analyze their performance relative to peers.
The dashboard integrates all gamification elements into a cohesive user experience.

\subsubsection{History Tracking}
\paragraph\ The platform maintains comprehensive records of all student activities and achievements.
The history system allows students to review their XP gains from each quest, track their progression over time, and understand the relationship between their efforts and outcomes.

\subsection{Faculty Management}

\subsubsection{Quest Configuration}
\paragraph\ Faculty members can create and manage quests through both individual interfaces and bulk Excel imports.
The system allows for dynamic quest addition throughout the term, enabling responsive curriculum adjustment based on student needs and learning objectives.

\subsubsection{Rank and XP Configuration}
\paragraph\ Instructors have comprehensive control over rank thresholds and XP distribution.
This flexibility enables adaptation to different course structures, student populations, and learning objectives while maintaining game balance and engagement.

\subsection{Assessment Integration}

\subsubsection{Continuous Assessment Connection}
\paragraph\ The gamification system integrates directly with academic assessment, allowing XP and rank progression to contribute to final course grades.
This connection ensures that game mechanics support rather than distract from educational objectives.

\subsubsection{Difficulty and Challenge Scaling}
\paragraph\ The platform incorporates mechanisms for adjusting challenge levels and providing appropriate difficulty progression.
This ensures that all students can find appropriate challenges while maintaining engagement across diverse skill levels and learning paces.
\end{comment}

\chapter{System Architecture and Design}
\section{Overview of LS Architecture}
\section{Technological Stack Selection}
\section{System Components and Modular Design}
\subsection{Student Interface and Experience}
\subsection{Teacher Dashboard and Analytics}
\subsection{Administrator Functions and Control}
\section{Integrating the Ontological Framework}
\section{Plugin Integration with Moodle}
\subsection{Concept Mapping}
\centering
\begin{center}
  \captionof{table}{Learning Scorecard vs Moodle Concept Mapping}
  \begin{tabular}{ccc}
    Concept            & Moodle                                                                & Feasibility \\ \hline
    \multicolumn{3}{c}{\textit{Academic}}                                                                    \\
    Course             & Course Categories                                                     & T           \\
    Curricular Unit    & Course                                                                & T           \\
    Student            & Users \& Roles                                                        & T           \\
    Teacher / Faculty  & Users \& Roles                                                        & T           \\
    Syllabus Contents  & Course Modules \& Tags                                                & T           \\
    Calendar           & Calendar Events                                                       & P           \\
    Timeline           & Activity Completion \& Progress                                       & P           \\
    \multicolumn{3}{c}{\textit{Gamification}}                                                                \\
    Experience Points  & Grade Items \& Custom Fields                                          & P           \\
    Ranks              & Custom Implementation                                                 & P           \\
    Quests             & Activities                                                            & T           \\
    Alliances          & Groups \& Cohorts                                                     & T           \\
    Guilds             & Groups                                                                & T           \\
    Badges             & Badges System                                                         & T           \\
    Trophies           & Badges System \& Custom Impl.                                         & P           \\
    Avatars            & User Profiles, Files \& Custom Impl.                                  & P           \\
    Leaderboards       & Gradebook \& Custom Views                                             & P           \\
    Last Chance System & Conditional Activities                                                & P
  \end{tabular}
  \caption*{T - Total Mapping  P - Partial Mapping}
\end{center}

\section{Database Design and Data Flow}
\section{Security and Scalability Considerations}

\chapter{Evaluation and Results}
\section{Demonstration of Key Features}
\section{Use Cases: Pilot Deployments and User Stories}
\section{Feedback and Empirical Results}
\section{Comparative Analysis with Previous Approaches}
\section{Discussion of Evaluation Metrics}

\chapter{Critical Analysis, Limitations, and Lessons Learned}
\section{Strengths and Achievements}
\section{Identified Limitations and Challenges}
\section{Reflections on the Development Process}
\section{Implications for Future Educational Technology}

\chapter{Conclusions and Future Directions}
\section{Summary of Contributions}
\section{Recommendations for Practice}
\section{Avenues for Further Research}

\renewcommand{\bibname}{References}

\def\bibindent{0.7cm}
\begin{thebibliography}{99\kern\bibindent}
  \makeatletter
  \let\old@biblabel\@biblabel
  \def\@biblabel#1{\old@biblabel{#1}\kern\bibindent}
  \let\old@bibitem\bibitem
  \def\bibitem#1{\old@bibitem{#1}\leavevmode\kern-\bibindent}
  \makeatother
  \makeatletter
  \renewcommand\@biblabel[1]{}
  \makeatother

  %\begin{thebibliography}{99}
  \bibitem{aka73} H. Akaike (1973), \textquotedblleft\ Information Theory as an
  Extension of the Maximum Likelihood Principle\textquotedblright, in B. N.
  Petrov, and F. Csaki, (Eds.), \textit{Second International Symposium on
  Information Theory}, Akademiai Kiado, Budapest, pp. 267--281.

  \bibitem{and10} D.T. Anderson, J.C. Bezdek, M. Popescu, and J.M. Keller
  (2010), \textquotedblleft\ Comparing Fuzzy, Probabilistic, and Possibilistic
  Partitions\textquotedblright, \textit{IEEE Transactions on Fuzzy Systems},
  18(5), 906--918.
\end{thebibliography}

\end{document}

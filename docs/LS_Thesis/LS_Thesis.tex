\documentclass[12pt,reqno, twoside]{amsbook}
%%%%%%%%%%%%%%%%%%%%%%%%%%%%%%%%%%%%%%%%%%%%%%%%%%%%%%%%%%%%%%%%%%%%%%%%%%%%%%%%%%%%%%%%%%%%%%%%%%%%%%%%%%%%%%%%%%%%%%%%%%%%%%%%%%%%%%%%%%%%%%%%%%%%%%%%%%%%%%%%%%%%%%%%%%%%%%%%%%%%%%%%%%%%%%%%%%%%%%%%%%%%%%%%%%%%%%%%%%%%%%%%%%%%%%%%%%%%%%%%%%%%%%%%%%%%
\usepackage{eurosym}
\usepackage{amsmath}
\usepackage{amssymb}
\usepackage{amsfonts}
\usepackage[onehalfspacing]{setspace}
\usepackage{chngcntr}
\usepackage{graphicx}
\usepackage[a4paper, margin=2.5cm]{geometry}
\usepackage[english]{babel}
\usepackage{fancyhdr}
\usepackage{titlesec}
\usepackage{enumitem}
\usepackage{etoolbox}
% you may need to uncomment the command below if you use eps format for figures
%\usepackage{epstopdf}

\makeatletter
\def
\section{\@startsection{section}{1}%
  \z@{.5\linespacing\@plus.7\linespacing}{.25\linespacing}%
{\normalfont\bfseries\flushleft}}
\def
\subsection{\@startsection{subsection}{2}%
  \z@{.5\linespacing\@plus.7\linespacing}{.25\linespacing}%
{\normalfont\bfseries\flushleft}}

\makeatother
\setcounter{MaxMatrixCols}{10}

\providecommand{\U}[1]{\protect\rule{.1in}{.1in}}
\theoremstyle{plain}
\newtheorem{acknowledgement}{Acknowledgement}
\newtheorem{algorithm}{Algorithm}[chapter]
\newtheorem{axiom}{Axiom}[chapter]
\newtheorem{case}{Case}[chapter]
\newtheorem{claim}{Claim}[chapter]
\newtheorem{conclusion}{Conclusion}[chapter]
\newtheorem{condition}{Condition}[chapter]
\newtheorem{conjecture}{Conjecture}[chapter]
\newtheorem{corollary}{Corollary}[chapter]
\newtheorem{criterion}{Criterion}[chapter]
\newtheorem{definition}{Definition}[chapter]
\newtheorem{example}{Example}[chapter]
\newtheorem{exercise}{Exercise}[chapter]
\newtheorem{lemma}{Lemma}[chapter]
\newtheorem{notation}{Notation}[chapter]
\newtheorem{problem}{Problem}[chapter]
\newtheorem{proposition}{Proposition}[chapter]
\newtheorem{remark}{Remark}[chapter]
\newtheorem{solution}{Solution}[chapter]
\newtheorem{summary}{Summary}[chapter]
\newtheorem{theorem}{Theorem}[chapter]
\numberwithin{equation}{chapter}

% Please write the references according to your school

\newenvironment{dedication}
{%\clearpage           % we want a new page
  \thispagestyle{empty}% no header and footer
  \vspace*{\stretch{1}}% some space at the top
  \itshape             % the text is in italics
  \raggedleft          % flush to the right margin
}
{\par % end the paragraph
  \vspace{\stretch{3}} % space at bottom is three times that at the top
  %\clearpage           % finish off the page
}
\numberwithin{section}{chapter}
\fancyhead{}
\fancyfoot{}
\pagestyle{fancy}
\fancyfoot[LE,RO]{\thepage}

\makeatletter
\def\ps@plain{\ps@empty
  \def\@evenfoot{%
    \normalfont\scriptsize
    \rlap{\thepage}\hfil
  }%
  \def\@oddfoot{%
    \normalfont\scriptsize \hfil
  \llap{\thepage}}%
}
\makeatother
\renewcommand{\headrulewidth}{0pt}
\renewcommand{\footrulewidth}{0pt}

\begin{document}
\frontmatter
\addtocontents{toc}{\setcounter{tocdepth}{2}}
\thispagestyle{empty}

\begin{dedication}%

  \begin{flushright}
    \textit{Write here your dedication!}
  \end{flushright}%
\end{dedication}

\setcounter{page}{1}
\pagenumbering{roman} %
\chapter*{Cover}

\chapter*{Acknowledgment}

Write here the acknowledgments and grants, if any

\chapter*{Resumo}

Write here your abstract in Portuguese

\chapter*{Abstract}

Write here your abstract in English

\tableofcontents

\mainmatter
\setcounter{page}{1} \pagenumbering{arabic}

\chapter{Introduction}
\noindent The digital transformation of educational institutions has created new opportunities for integrating analytics and gamification in order to improve learning outcomes.
The Learning Scorecard (LS) platform embodies this approach, merging business intelligence with educational strategies to monitor and enhance student engagement and performance.

This thesis presents the development of the Learning Scorecard v2.5: a
prototype plugin that integrates some of the major features of the previous LS
iterations with the Moodle Learning Management System Service API.\@
\section{Background and Rationale}
\section{Statement of the Problem}
\section{Aims and Objectives}
\section{Methodological Approach}
\section{Contribution to Knowledge}
\section{Structure of the Thesis}

\chapter{Theoretical and Methodological Foundations}
\section{Gamification and Motivation in Education}
\subsection{Theoretical Perspectives on Gamification}
\subsection{Gamification and Student Engagement}
\section{Learning Analytics and Educational Data Mining}
\subsection{Defining Learning Analytics}
\subsection{Data-Driven Decision Making in Higher Education}
\section{Agile Methodology in Educational Software Development}
\subsection{Principles of Agile Methodology}
\subsection{Agile Documentation Practices}

\chapter{The Learning Scorecard: Concept, Evolution, and Requirements}
\section{Genesis and Conceptualization of the Learning Scorecard}
\section{Evolution Through Iterative Development}
\subsection{Historical Overview of Versions}
\subsection{Lessons Learned from Prior Iterations}
\section{Defining Functional and Non-Functional Requirements}

\chapter{System Architecture and Design}
\section{Overview of LS 2.5 Architecture}
\section{Technological Stack Selection}
\section{System Components and Modular Design}
\subsection{Student Interface and Experience}
\subsection{Teacher Dashboard and Analytics}
\subsection{Administrator Functions and Control}
\section{Integrating the Ontological Framework}
\section{Plugin Integration with Moodle}
\section{Database Design and Data Flow}
\section{Security and Scalability Considerations}

\chapter{Implementation and Agile Development Process}
\section{Adoption of Agile Methodology}
\subsection{Sprint Planning and Execution}
\subsection{Continuous Feedback and Iterative Refinement}
\section{Key Implementation Milestones}
\section{User Interface Design and Usability}
\section{Addressing Implementation Challenges}
\section{Documentation as a Living Artifact}

\chapter{Evaluation and Results}
\section{Demonstration of Key Features}
\section{Use Cases: Pilot Deployments and User Stories}
\section{Feedback and Empirical Results}
\section{Comparative Analysis with Previous Approaches}
\section{Discussion of Evaluation Metrics}

\chapter{Critical Analysis, Limitations, and Lessons Learned}
\section{Strengths and Achievements}
\section{Identified Limitations and Challenges}
\section{Reflections on the Development Process}
\section{Implications for Future Educational Technology}

\chapter{Conclusions and Future Directions}
\section{Summary of Contributions}
\section{Recommendations for Practice}
\section{Avenues for Further Research}
\section{Final Remarks}

\renewcommand{\bibname}{References}

\def\bibindent{0.7cm}
\begin{thebibliography}{99\kern\bibindent}
  \makeatletter
  \let\old@biblabel\@biblabel
  \def\@biblabel#1{\old@biblabel{#1}\kern\bibindent}
  \let\old@bibitem\bibitem
  \def\bibitem#1{\old@bibitem{#1}\leavevmode\kern-\bibindent}
  \makeatother
  \makeatletter
  \renewcommand\@biblabel[1]{}
  \makeatother

  %\begin{thebibliography}{99}
  \bibitem{aka73} H. Akaike (1973), \textquotedblleft Information Theory as an
  Extension of the Maximum Likelihood Principle\textquotedblright, in B. N.
  Petrov, and F. Csaki, (Eds.), \textit{Second International Symposium on
  Information Theory}, Akademiai Kiado, Budapest, pp. 267--281.

  \bibitem{and10} D.T. Anderson, J.C. Bezdek, M. Popescu, and J.M. Keller
  (2010), \textquotedblleft Comparing Fuzzy, Probabilistic, and Possibilistic
  Partitions\textquotedblright, \textit{IEEE Transactions on Fuzzy Systems},
  18(5), 906--918.
\end{thebibliography}

\end{document}

\documentclass[12pt,reqno, twoside]{amsbook}
%%%%%%%%%%%%%%%%%%%%%%%%%%%%%%%%%%%%%%%%%%%%%%%%%%%%%%%%%%%%%%%%%%%%%%%%%%%%%%%%%%%%%%%%%%%%%%%%%%%%%%%%%%%%%%%%%%%%%%%%%%%%%%%%%%%%%%%%%%%%%%%%%%%%%%%%%%%%%%%%%%%%%%%%%%%%%%%%%%%%%%%%%%%%%%%%%%%%%%%%%%%%%%%%%%%%%%%%%%%%%%%%%%%%%%%%%%%%%%%%%%%%%%%%%%%%
\usepackage{eurosym}
\usepackage{amsmath}
\usepackage{amssymb}
\usepackage{amsfonts}
\usepackage[onehalfspacing]{setspace}
\usepackage{chngcntr}
\usepackage{graphicx}
\usepackage[a4paper, margin=2.5cm]{geometry}
\usepackage[english]{babel}
\usepackage{fancyhdr}
\usepackage{titlesec}
\usepackage{enumitem}
\usepackage{etoolbox}
% you may need to uncomment the command below if you use eps format for figures
%\usepackage{epstopdf}

\makeatletter
\def
\section{\@startsection{section}{1}%
  \z@{.5\linespacing\@plus.7\linespacing}{.25\linespacing}%
{\normalfont\bfseries\flushleft}}
\def
\subsection{\@startsection{subsection}{2}%
  \z@{.5\linespacing\@plus.7\linespacing}{.25\linespacing}%
{\normalfont\bfseries\flushleft}}

\makeatother
\setcounter{MaxMatrixCols}{10}

\providecommand{\U}[1]{\protect\rule{.1in}{.1in}}
\theoremstyle{plain}
\newtheorem{acknowledgement}{Acknowledgement}
\newtheorem{algorithm}{Algorithm}[chapter]
\newtheorem{axiom}{Axiom}[chapter]
\newtheorem{case}{Case}[chapter]
\newtheorem{claim}{Claim}[chapter]
\newtheorem{conclusion}{Conclusion}[chapter]
\newtheorem{condition}{Condition}[chapter]
\newtheorem{conjecture}{Conjecture}[chapter]
\newtheorem{corollary}{Corollary}[chapter]
\newtheorem{criterion}{Criterion}[chapter]
\newtheorem{definition}{Definition}[chapter]
\newtheorem{example}{Example}[chapter]
\newtheorem{exercise}{Exercise}[chapter]
\newtheorem{lemma}{Lemma}[chapter]
\newtheorem{notation}{Notation}[chapter]
\newtheorem{problem}{Problem}[chapter]
\newtheorem{proposition}{Proposition}[chapter]
\newtheorem{remark}{Remark}[chapter]
\newtheorem{solution}{Solution}[chapter]
\newtheorem{summary}{Summary}[chapter]
\newtheorem{theorem}{Theorem}[chapter]
\numberwithin{equation}{chapter}

% Please write the references according to your school

\newenvironment{dedication}
{%\clearpage           % we want a new page
  \thispagestyle{empty}% no header and footer
  \vspace*{\stretch{1}}% some space at the top
  \itshape             % the text is in italics
  \raggedleft          % flush to the right margin
}
{\par % end the paragraph
  \vspace{\stretch{3}} % space at bottom is three times that at the top
  %\clearpage           % finish off the page
}
\numberwithin{section}{chapter}
\fancyhead{}
\fancyfoot{}
\pagestyle{fancy}
\fancyfoot[LE,RO]{\thepage}

\makeatletter
\def\ps@plain{\ps@empty
  \def\@evenfoot{%
    \normalfont\scriptsize
    \rlap{\thepage}\hfil
  }%
  \def\@oddfoot{%
    \normalfont\scriptsize \hfil
  \llap{\thepage}}%
}
\makeatother
\renewcommand{\headrulewidth}{0pt}
\renewcommand{\footrulewidth}{0pt}

\begin{document}
\frontmatter
\addtocontents{toc}{\setcounter{tocdepth}{2}}
\thispagestyle{empty}

\begin{dedication}%

  \begin{flushright}
    \textit{Write here your dedication!}
  \end{flushright}%
\end{dedication}

\setcounter{page}{1}
\pagenumbering{roman} %
\chapter*{Cover}

\chapter*{Acknowledgment}

Write here the acknowledgments and grants, if any

\chapter*{Resumo}

Write here your abstract in Portuguese

\chapter*{Abstract}

Write here your abstract in English

\tableofcontents

\mainmatter

\hyphenation{wri-te he-re pro-per hy-phe-ni-za-tion}%

\setcounter{page}{1} \pagenumbering{arabic}

\chapter{Introduction}
\noindent You must use $\backslash$noindent at the beginning of the first paragraph in sections and subsections.
\section{Motivation}
\section{Methodology}
\section{Tecnology}
\newpage
We add a page brake to show that the even page number appears on the left

\chapter{Gamification in Higher Education}
\section{One Section}
\section{Another Section}

\chapter{An overview of the Learning Scorecard}
\section{Retrospective \& Starting Point}
\subsection{Problems of previous iterations}
\section{Moodle Implementation}

\chapter{Learning Scorecard 2.5}
\section{The MVP}
\section{Database Documentation}
\section{Interfaces}

\chapter{Conclusions}
\section{Critical Analysis}
\section{Advantages vs Disadvantages}

\renewcommand{\bibname}{References}

\def\bibindent{0.7cm}
\begin{thebibliography}{99\kern\bibindent}
  \makeatletter
  \let\old@biblabel\@biblabel
  \def\@biblabel#1{\old@biblabel{#1}\kern\bibindent}
  \let\old@bibitem\bibitem
  \def\bibitem#1{\old@bibitem{#1}\leavevmode\kern-\bibindent}
  \makeatother
  \makeatletter
  \renewcommand\@biblabel[1]{}
  \makeatother

  %\begin{thebibliography}{99}
  \bibitem{aka73} H. Akaike (1973), \textquotedblleft Information Theory as an
  Extension of the Maximum Likelihood Principle\textquotedblright, in B. N.
  Petrov, and F. Csaki, (Eds.), \textit{Second International Symposium on
  Information Theory}, Akademiai Kiado, Budapest, pp. 267--281.

  \bibitem{and10} D.T. Anderson, J.C. Bezdek, M. Popescu, and J.M. Keller
  (2010), \textquotedblleft Comparing Fuzzy, Probabilistic, and Possibilistic
  Partitions\textquotedblright, \textit{IEEE Transactions on Fuzzy Systems},
  18(5), 906--918.
\end{thebibliography}

\end{document}
